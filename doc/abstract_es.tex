\section{Resumen}
\label{sec:abstract_es}

Empresas financieras en el mundo comenzaron hace ya algunos años a focalizar sus esfuerzos e inversiones en fondos basados en métodos cuantitativos implementados con algoritmos y corriendo en servidores que ejecutan compras y ventas automáticamente. Distinguimos el trading de alta frecuencia del trading basado en aprendizaje de máquina, el cual será objeto de estudio en este trabajo.

En particular, el presente trabajo muestra el desarrollo de una estrategia de compraventa de Bitcoins basada en técnicas de aprendizaje automático. Se propone una implementación del proceso presentado en el libro de Lopez de Prado \cite{lopez_de_prado}. Al proceso anterior se lo modifica para poder incorporar un modelo de momentum sobre Bitcoins y se realiza un estudio sobre los features que permitirán mejorar un modelo secundario (a aprender) para dimensionar los tamaños de las posiciones que la estrategia fundamental (de momentum) proponga. Los features a analizar y comparar van desde métricas financieras del activo subyacente, métricas propias de la tecnología de blockchain y Bitcoin como criptoactivo para terminar con métricas sociales que den información sobre interés y animosidad.

El proceso planteado en \cite{lopez_de_prado} e implementado en este trabajo busca hacer un cuidado riguroso del dataset y los modelos empleados así como posterior backtesting de la estrategia. Este informe detalla los detalles de la implementación tanto a nivel estadístico, algorítmico y de dominio (criptoactivos). A partir de los resultados obtenidos en cada etapa del proceso y finalmente en backtesting podremos analizar el proceso de generación de estrategias sino que también comparar algunas para entender el proceso de selección. En particular, es de relevancia en este trabajo utilizar en contraste índices de microestructura como SADF (Supremum Augmented Dickey Fuller) en comparación y conjunto con los índices sociales.   
